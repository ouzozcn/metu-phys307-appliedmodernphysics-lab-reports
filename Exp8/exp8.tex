\documentclass[a4paper,12pt]{article}
\usepackage{graphicx}
\usepackage{titlesec}
\usepackage[utf8]{inputenc}
\usepackage{xcolor}
\usepackage{fancyhdr}
\usepackage{lipsum}
\usepackage{caption}

\renewcommand{\headrulewidth}{0pt}
\fancyhead[C]{}
\fancyhead[C]{
	\includegraphics[width=4cm]{metu}
}
\pagestyle{plain}

%opening
\title{Middle East Technical University\\Department of Physics\\\textbf{PHYS307 Applied Modern Physics}}
\author{Oğuzhan ÖZCAN\\1852334}
\date{}
\clearpage
\thispagestyle{empty}
\providecommand{\groupmember}[1]{\textbf{Group Members:} }
\providecommand{\expdate}[1]{\textbf{Experiment Date:} }
\providecommand{\repdate}[1]{\textbf{Report Submit Date:} }
\providecommand{\expname}[1]{\textbf{Exp. MP-SA The Energy Spectra of Alpha Particles} }


\usepackage[a4paper,%
left=0.5in,right=0.5in,top=0.5in,bottom=0.8in,%
footskip=.25in]{geometry}
%\topmargin -4.5cm
%\oddsidemargin 0.2cm
%\textwidth 16cm %
%\textheight 21cm%
%\footskip 1.0cm%




\begin{document}
\pagenumbering{gobble}
\maketitle

\thispagestyle{fancy}

%%%%%%%%%%%%%%%%%%%%%%%%%%%%%%%%%%%%%%%%%%%%%%%%%%%%%
\noindent\rule{18.4cm}{0.8pt}
\begin{center}
	\expname{arg1}{}
\end{center}
\groupmember{arg1}{Cem MADEN, İrem KÜL, Deniz AKYÜREK}\\
\expdate{November 6, 2015}{November 27, 2015}\\
\repdate{arg1}{December 4, 2015}\\
\noindent\rule{18.4cm}{0.8pt}\\\\
%%%%%%%%%%%%%%%%%%%%%%%%%%%%%%%%%%%%%%%%%%%%%%%%%%%%%
\begin{table}[h!]
	\begin{center}
	\begin{tabular}{|c|c|c|}
	\hline Peak (channel no) & Source of $\alpha$-particles & Energy (in keV) \\ 
	\hline 76 & $^{226}$Ra & 4780 \\ 
	\hline 118 & $^{222}$Rn & 5551 \\ 
	\hline 146 & $^{218}$Po & 6065 \\ 
	\hline 234 & $^{214}$Po & 7680 \\ 
	\hline 
\end{tabular}
\caption{The resolved peaks in the spectrum of $^{226}$Ra in the vacuum}
\end{center}
\end{table}

\begin{table}[h!]
\begin{center}
		\begin{tabular}{|c|c|}
	\hline Peak (channel no) & Energy (in keV) \\ 
	\hline 52 & 4340 \\ 
	\hline 100 & 5221 \\ 
	\hline 140 & 5954 \\ 
	\hline 230 & 7606 \\ 
	\hline 
	\end{tabular}
	\caption{The resolved peaks in the spectrum of $^{226}$Ra in the air}
\end{center}
\end{table}

\begin{table}[h!]
	\begin{center}
		\begin{tabular}{|c|c|c|c|}
	\hline Peak (channel no) & Total Pulses & Energy (in keV) & Source of $\alpha$-particle \\ 
	\hline 151 & 4280 & 6156 & $^{244}$Cm \\ 
	\hline 
	\end{tabular}
	\caption{Data for unknown $\alpha$-source in the vacuum} 
	\end{center}
\end{table}
\newpage
\textbf{1. Calculate the activity of the unknown $\alpha$-source.}\\\\
\begin{equation}
Activity(\alpha/s)=(\frac{\sum\alpha}{t})(\frac{4\pi S^{2}}{A})
\end{equation}
where S=0.02 m, A=4.0$\times 10^{-6}$m$^{2}$, $\sum \alpha = 4280$ and t=300 secs.
\begin{equation}
Activity(\alpha/s)=(\frac{4280}{300})(\frac{4\cdot \pi \cdot0.02^{2}}{4.0\times 10^{-6}})
\end{equation}
\begin{equation}
Activity(\alpha/s)=17928 decay/s
\end{equation}
\begin{equation}
Activity(\alpha/s)=0.48 \mu Ci
\end{equation}
\textbf{2. Write at least five different reasons for observing only a few peaks rather than nine peaks in the spectrum of $^{226}$Ra in the vacuum and make suggestions to improve the resolving of the peaks in the spectrum with the present system.}\\\\
1- We could not vacuumed the detector 100\%.\\
2- Noise and vibration\\
3- $^{216}$At, $^{218}$Rn, $^{210}$Tl and $^{206}$Tl in the decay sequence of $^{226}$Ra only occur with very low activities.\\
4- Loss of radon daughters [1].\\
5- The detector does not have high sensivity.\\
6- In the experiment we used five different experimental setups and each one have it's own calculation error.\\
To be more accurate in the experiment my suggestion is splitting vacuum chamber and other experiment setups because noise and vibration are very effective to distort the data. Using an high tech detector will give more accurate solutions. In laboratory, other radioactive decays may cause to distortion in data. Some researchers state that a calibration before the experiment may reduce to error in experiment. $^{210}$Po, $^{238}$U, $^{238}$Pu, $^{239}$Pu and $^{241}$Am  is strongly preferred over pulser-calibration techniques [2].\\\\
\textbf{3. Describe the physical significance of any peak on energy spectra of $\alpha$-particles.}\\\\
Each peak is a $\alpha$ decay of $^{226}$Ra. In each peak $^{226}$Ra is turned to his isotopes in the order of as stated in Table 1. By using these peaks we can calculate the energy of $\alpha$ decays, we can detect the type of radioactive isotopes.\\\\
\textbf{Discussion and Conclusion }\\\\
In this experiment we observe the $\alpha$ decay of $^{226}$Ra in both vacuum and air. After this experiment we can realize the behaviour of $\alpha$ decays in different environment. Another significant observation in this experiment is the decays of unknown sample. By using decay rate of unknown sample we tried to predict that name of radioactive isotope. In this experiment we have very low percantage error. For instance for $^{222}$Rn which is took in vacuumed chamber:
\begin{equation}
Per. Error = \frac{|5480-5550|}{5480}\times 100 \%
\end{equation}
\begin{equation}
Per. Error = 1.3 \%
\end{equation}
\newpage
\textbf{References}\\\\
$[1]$ \textit{Alpha-, Beta-, and Gamma-Ray Spectroscopy},
Ed.: Kai Siegbahn, Vol. 1, North-Holland Publ. Co., Amsterdam, New York (1968)\\\\
$[2]$ \textit{Nuclear Spectroscopy and Reactions,}
Ed.: J. Cerny, Academic Press, New York and London (1974)


 











































































































































\end{document}
